%!TEX root = ../thesis.tex
\chapter{Background}
\label{ch:background}


\section{Hexapod Robots}
Hexapods are 6-legged, insect-like Robots.
Compared to wheeled or two-legged robots they are much more stable on uneven terrain and can navigate more precisely.
Each leg of a hexapod typically consists of 3 segments named coxa, femur and tibia, just like their biological counterparts.
The segments are connected to each other via 3 2-DoF joints, each actuated by an electric motor.
\todo[color=Aquamarine1]{Add image of hexapod(and maybe of insect)}


\section{MATLAB}
\subsection{Simulink}
\textit{Simulink\textsuperscript{\textregistered}} is a \textit{MATLAB\textsuperscript{\textregistered}}-based graphical block-diagramming tool developed by the company \textit{MathWorks\textsuperscript{\textregistered}} intended for designing and simulating dynamical systems.
It provides an expansive library of continuous and discrete blocks, which enables the user to create systems from various domains.
The creation of a Simulink model is done by connecting blocks via edges in a GUI.
The blocks transform input data provided by the connected edges, the edges forward the data to other blocks connected downstream.
Blocks can be put into subsystems, creating different levels of abstraction.
For reuse, subsystems can be placed in libraries, so that all linked subsystems can be updated in one place.
Once a system is build it can be simulated using on of the many solvers provided.

\subsection{Simscape}
\textit{Simscape\textsuperscript{\texttrademark}} is a block library developed by \textit{MathWorks\textsuperscript{\textregistered}} with which it is possible to model physical system within the Simulink environment.
It is capable of modeling and simulating systems such as electric circuits, hydraulics or rigid body interactions.
Simscape also provides a MATLAB based language to enable text-based development of custom components.
 



\section{Inverse Kinematics}
Inverse Kinematics(IK) is a term predominately used in robotics or computer graphics.
It describes the process of calculating the joint angles needed to place the end of a kinematic chain, such as a robotic manipulator, at a given position and orientation.
There are two main methods how to calculate these angles, analytical and numerical.
Analytical solvers are based on trigonometric equations and provide exact solutions, but can only be implemented for manipulators with less than 4(?) DOF, otherwise there are infinite solutions.
Numerical solvers use iterate approaches to closer and closer approximate the correct solutions.
Analytical solvers are generally much faster, though the iterative approach of numerical solvers enables them to be used on all inverse kinematics problems without limitation.





\section{PID Controller}
A \textbf{p}roportional-\textbf{i}ntegral-\textbf{d}erivative controller(PID controller) is type of controller widely used in industry.
It controls a so called "Control Variable" by listening to the feedback it gets from the environment and adjusts its output accordingly.
 


\section{Reinforcement Learning}
