%!TEX root = ../thesis.tex
\chapter{Conclusion and Outlook}
\label{ch:conclusion}

In this concluding chapter, we summarize our key finding and contributions.
We discuss the value of the developed model as a foundation for future research and highlight the insights we gained into the task of applying RL to the problem of leg coordination.
Lastly, we provide an outlook toward potentially interesting future research topics, expanding on the foundations established in our work.

\section{Conclusions}

MATLAB Simulink proved to be a versatile and reliable platform for the modelling and simulation of robotic systems.
The development of the hexapod model was greatly sped up by the underlying framework and the utilisation of the provided toolboxes.
The ability to write MATLAB-scripts to automate several task, like the initialisation of the hexapod model or the setup of the RL agents and surrounding infrastructure, significantly increased our productivity.
We are confident that the developed model can serve as a platform for future research, as it utilizes a modular architecture, can easily be adapted to changing requirements and is based on the versatile MATLAB environment.
The model also proved surprisingly robust in the face of constant twitching caused by exploring RL agents.

Another benefit the use of MATLAB provided was the \textit{Parallel Computing Toolbox}. 
By being able to run 8-16 agents in parallel, one per processor core or multiple in separate threads, we were able to greatly increase the number of RL training episodes and could thus faster reach conclusions on the examined agents.
As we were unsuccessful in creating an agent with an emergent gait, we can not speak to the effectiveness of continuing the training of already performant agents for even longer periods.

As to the reasons for why the RL approach did not yield satisfying results, we have several potential explanations:
\begin{itemize}
	\item \textbf{Inadequate reward function: } As we already mentioned, defining a reward function which accurately judges an agent's performance is a difficult task, requiring expertise.
	Although we based our reward function on existing literature \parencite{trotta2022walking, schilling2021decentralized}, refined it's weight over several iterations and thoroughly tested it judgement on the static gait patterns, we cannot rule out a flaw in the function, which might lead to exploitation or loss of relevant information.
	
	\item \textbf{RL algorithm configuration: } Over the course of this work we experimented with many different RL parameters, many of which have already been successfully applied by other authors \parencite{matlabDDPGExample, matlabPPOExample, trotta2022walking}.
	Even so, there does not exist any literature on the same problem of leg coordination given a fixed leg trajectory and thus the possibility exists that we did not discover a performant set of parameters.
	As the learning ability of both policy optimization methods (DDPG, PPO) was observed to be about equal, we are however confident that the chosen algorithms are not the cause of the observed problems.
	
	\item \textbf{Continued Development: } As we continued to develop and improve both the locomotion controller and the RL setup while we were already training RL agents, there does not exist a fixed, final state of the underlying model, on which all training was based.
	The result of which is a limited comparability of the the trained agents.
	As our main goal was to provide a well-performing hexapod model, we decided to rather fix mistakes we discovered, sacrificing comparability.
	
\end{itemize}
We are confident that, given more time to carefully tune parameters and reward, a RL agent is capable of finding an efficient gait pattern given the implemented setup.

Our suggestions for future researchers attempting to apply RL to the leg coordination problems are the following:
\begin{enumerate}
	\item Keep the underlying model fixed during experimentation with different RL agents. 
	Changes to the model make it difficult to compare the performance of RL parameter sets.
	The hexapod model should be meticulously tested prior to the application of RL.
	
	\item Automate the process of running several agents with different parameter sets. 
	By automating this task, the speed of exploration in the parameter space could be greatly increased.
	
	\item Meticulously collect the data of each agent's performance to be able to more quickly discover trends.
	
\end{enumerate}

Given that the primary objective of this thesis was the creation of a functional and resilient hexapod model, serving as a foundational platform for future research, we consider our efforts a success.
In addition to the hexapod model itself, we successfully designed a basic locomotion controller and established a framework for applying RL to the hexapod system, offering potential starting point for future expansions.
In summary, the development of the hexapod model and locomotion controller can be deemed successful, while the approach to solving the RL leg coordination problem presents an ongoing challenge to which further research has to be applied.


\section{Outlook - Future Work}

In this last section, we draw our work to a close by suggesting possible future expansions to the model and more broadly point of future research interest.
As we also only implemented a rudimentary locomotion controller capable of navigation on flat terrain, expanding the controller's capabilities to more complex, uneven terrain would be a valuable research target.
Since the goal of successfully applying RL to the problem of leg coordination was not reached in completion, it might also be appealing, as a future work, to focus solely on this problem.
This could include a deeper investigation into reward function definitions and the task of tuning the parameters of a RL algorithm.
Looking further ahead, a substantial research goal could involve applying RL to learn hexapod locomotion from scratch, granting the agent complete control over all 18 DoF.\\
In conclusion, while our current work lays a strong foundation on which further works can be based, it also reveals a range of unexpected difficulties to be explored in the future..


\begin{comment}

\begin{itemize}
	
	\item Controller improvements:
		\subitem Curve walking
		\subitem Adaptation to uneven terrain
		\subitem 
	
	\item Expand Reinforcement Learning from just leg coordination towards controlling the whole movement process
		\subitem Improve reward definition/ terminate episodes in which no progress seems to be made after some time(such as in \cite{lillicrap2015continuous})
		\subitem Learn walking from the ground up, meaning the agent controls each of the 18 joints (maybe take steps in between)
		\subitem
	
	\item Use other MATLAB Tools such as Hardware Co-Simulation to test controller on real robot
	
	\item 
	
\end{itemize}
\end{comment}