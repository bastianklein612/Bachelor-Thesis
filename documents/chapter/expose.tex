%!TEX root = ../thesis.tex
\chapter{Exposé}
\label{ch:expose}



\section{Einleitung}
Der Einsatz von Simulationen spielt in der Erforschung und Entwicklung von Robotern eine zunehmend wichtige Rolle\parencite{afzal2020study}. 
Sie bieten eine kostengünstige und effiziente Methode, das Verhalten von Robotern schon vor der physischen Umsetzung untersuchen, analysieren und optimieren zu können\parencite{de2019analysis}. 
Auch eine Vielzahl von Unternehmen, welche auf dem Gebiet der Robotik tätig sind, nutzten die Technologie ausgiebig im Entwicklungsprozess ihrer Systeme, so beispielsweise Boston Dynamics\parencite{BostonDynamicsSimulation}, Tesla Inc.\parencite{TeslaAiDay2022} oder Kuka Robotics\parencite{KukaSim}.

Auf Grund ihrer hohen Beweglichkeit sowie Vielseitigkeit sind Hexapods in der Robotik von besonderen Interesse. 
Verglichen mit beräderten oder zweibeinigen Robotern können sie sich zuverlässiger auf unebenem Gelände bewegen und auch komplexe Hindernisse überwinden\parencite{barai2013smart, atifystructure}.
Die Simulation eines solchen Hexapods ermöglicht es sein Verhalten in verschiedensten simulierten Umgebungen zu beobachten und erproben.

\section{Problem}

Die Entwicklung von Robotern und ihren Bewegungsabläufen ist zeitaufwändig und mit hohen Kosten verbunden\parencite{ahmadian2005model}.
Wird die entwickelte Software direkt in der Realität getestet, so besteht die Gefahr der Beschädigung eines der oft wenigen und teuren Testexemplare.
Dies kann den Entwicklungsprozess verzögern und zusätzliche Kosten generieren.
Auch ist ein iterativer Optimierungsprozess, welcher ausschließlich in einer realen Umgebung stattfindet langsamer, als wenn dieser mit Hilfe von Simulationen unterstützt wird.
Die Steuerungssoftware muss für jede Anpassung neu auf die Hardware übertragen werden und Testaufbau sowie Roboter müssen manuell wieder in den Ausgangszustand versetzt werden.
Ebenfalls kann es schwierig sein fehlerhaftes Verhalten, welches beim Test mit einem realen System beobachtet wurde, zu reproduzieren, ohne die Möglichkeit einer exakten Wiederherstellung des Testaufbaus.

Natürlich ist es letztendlich wichtig, den entwickelten Controller auch in der echten Welt zu testen, damit sichergestellt werden kann, dass dieser auch unter den tatsächlichen physikalischen Begebenheiten zuverlässig funktioniert.
Die Vorteile welche eine Simulationsumgebung mit sich bringt sollten jedoch keinesfalls übersehen werden, da diese für Entwickler und Forscher ein mächtiges Werkzeug darstellen kann um vor der realen Erprobung erste Tests durchzuführen, Fehler aufzudecken und den Controller zu optimieren.
%Um die Erforschung von Controllern zu Bewegungssteuerung möglichst effektiv vorantreiben zu können ist es wichtig, leistungsstarke Simulationssoftware in den Entwicklungsprozess einzubeziehen.
%Es wird ein wesentlich  kleinschrittiger Prozess ermöglicht, da jede noch so kleine Änderung direkt und ohne großen Mehraufwand getestet werden kann.

%Da auf Grund von Materialkosten, Konstruktions- und Wartungsaufwand reale Roboter oft diejenige Komponente eines Projektes sind, welche beim Erproben der neu entwickelten Software der größte zeitliche "bottleneck" sind, stellt das direkte Erproben komplexer Aufgaben in der realen Welt(z.B. Navigation auf unebenem Gelände)  ein Risiko dar.

%Zur Entwicklung des Controllers und der Simulation desselben auf einem Hexapod-Modell wird in dieser Arbeit das MATLAB Tool Simulink in Verbindung mit der Simulation-Toolbox SimScape verwendet werden.
%Simulink ist eine weit verbreitete Software zur Entwicklung und Simulation von komplexen dynamischen Systemen.
%Durch die Popularität der Software ist mit der Zeit eine Vielzahl von domänenspezifischen Bibliotheken(Toolboxes) entstanden, welche die einfache Modellierung verschiedenster Modelle ermöglichen.
%Der block-basierte Ansatz von Simulink ist dabei besonders hervorzuheben, da es dieser ermöglicht sehr modulare Systeme zu erzeugen, wodurch bei Anpassungen nur einzelne Subsysteme ausgetauscht werden, anstatt Änderungen am gesamten Modell vornehmen zu müssen.

\section{Ziel}
Das von uns im Rahmen dieser Bachelorarbeit angestrebte Ziel ist es, die Möglichkeiten von MATLAB Simulink zur Modellierung und Simulation komplexer dynamischer Systeme zu nutzen, um ein Modell eines Hexapod-Roboters sowie dessen Steuerung zu entwickeln.

Das Modell muss die kinematischen Eigenschaften des Roboters und der Umgebung in welcher er sich bewegt berücksichtigen.
Es soll eine realistische Simulation des Roboterverhaltens ermöglichen und dabei Informationen bezüglich der Positionierung des Modells bereitstellen.
Diese Daten sind entscheidend für die Steuerung des Roboters und müssen für den Controller jederzeit zur Verfügung stehen.
Der Bewegungscontroller des Hexapods sollte in der Lage sein, den Roboter zuverlässig und präzise zu steuern.
Dies beinhaltet die Bewegung des einzelnen Beins, die Koordination der Beine untereinander sowie das Gleichgewicht des Gesamtsystems aufrecht zu erhalten.
Die entwickelte Lösung soll den Roboter in verschiedenen Gangarten geradlinig in der Simulationsumgebung bewegen und auch kleine Hindernisse überwinden können, ohne dass das System aus dem Gleichgewicht gebracht wird.
Dabei darf die Steuerung nur auf vom Modell bereitgestellte Daten zugreifen, wie dies auch bei einem realen Roboter mittels Sensoren der Fall wäre.
Es sollte zusätzlich möglich sein, Module des Bewegungscontrollers anzupassen oder Eigenschaften des Modells zu ändern, ohne den gesamten Roboter neu konfigurieren zu müssen.
Nachdem ein robustes Modell und ein zuverlässiger Controller entwickelt wurden liegt der Fokus darauf, es dem Roboter mit Hilfe von maschinellem Lernen zu ermöglichen, die Koordination der Beine selbständig zu erlernen.
Es wäre besonders interessant festzustellen, ob schon bekannte Gangarten erkennbar werden oder ganz neue Formen der Bewegung entstehen.


\section{Ziel}
Das angestrebte Ziel dieser Bachelorarbeit ist es, einen Hexapod-Roboter und dessen Steuerung innerhalb der MATLAB Simulink Umgebung zu entwickeln.
Konkret soll dies die Modellierung des physischen Modells mit Hilfe von SimScape, den Aufbau eines Controllers für die Bewegungssteuerung des Roboters sowie das Erlernen der Koordination der Beine untereinander mittels Reinforcement Learning beinhalten.

\begin{enumerate}
	
	
	\item \textbf{Selbstständiges Erlernen der Beinkoordination:} Durch Interaktion mit der Simulationsumgebung soll das Modell selbstständig die Koordination der 6 Beine erlernen, die Beinbewegung selbst bleibe dabei vorgegeben.
	Modelliert werden soll der Lernprozess mit Hilfe der Reinforcement Learning Toolbox von Simulink.
	
\end{enumerate}

\section{Lösungsansatz}
Zu Beginn der Arbeit wird ein detailliertes, physikalisch korrektes Modell des Hexapods mithilfe der Simulink-Toolbox Simscape entwickelt, als Vorlage dient hier der \emph{PhantomX}-Hexapod von \emph{Trossen Robotics}.
Jedes der 6 Beine besteht aus 3 Gelenken mit je einem Aktuator, eine Architektur welche weit verbreitet ist und sich bewährt hat.
Das Modell sollte möglichst gut parametrisiert werden, damit Eigenschaften wie Gewicht, Reibungskoeffizienten, Drehmomente, etc. der einzelnen Komponenten auch im nach hinein noch einfach angepasst werden können.
Jeder der Aktuatoren muss einzeln ansprechbar sein und die Gelenke müssen Sensordaten bezüglich Position (und Beschleunigung) liefern.
Diese Daten informieren den im nächsten Schritt zu implementierenden Controller über die Lage (und Dynamik) des Roboters.
Neben dem eigentlichen Modell des Hexapods wird zusätzlich noch eine einfache Testumgebung erbaut, in welcher das Robotermodell im weiteren Entwicklungsprozess ausgiebig getestet werden kann.

Nachdem die Entwicklung des SimScape-Modells abgeschlossen ist, wird mit der Implementierung einer kontrollierten Steuerung des Roboters in Simulink begonnen.
Der Fokus liegt dabei zunächst auf der Beschreibung des Bewegungsablaufs eines einzelnen Beins.
Sobald dieser Ablauf, bestehend aus Schwung- und Standphase, zuverlässig funktioniert, wird er auf alle 6 Beine ausgeweitet.
Hierbei spielt die Koordination der Beine eine entscheidende Rolle, unabhängig davon, ob diese zentral oder verteilt ausgelegt wird.
Grundlegend ist es natürlich auch, dass der Roboter das Gleichgewicht hält und robust gegenüber Störungen im Bewegungsablauf ist.
Um das SimScape-Modell ausgiebig zu testen, ist es möglich, mehrere verschiedene Gangarten wie den Tripod-, Tetrapod- und Wave-Gait abzubilden.
%evt. könnte auch eine dezentrale Architektur wie z.B. die von Schilling et al. abgebildet werden (?).

Basierend auf den Vorarbeiten zur Modellierung und den verschiedenen Gangarten soll anschließend die Koordination der Beine mithilfe eines Reinforcement Learning-Modells selbst erlernt werden.
Dieser Lernprozess wird ebenfalls innerhalb der Simulink-Umgebung durch die RL-Toolbox realisiert.
Im Gegensatz zu den bisher erzeugten Gangarten wird hier nicht vorgegeben werden, welche Beine ihren Bewegung zu welchem Zeitpunkt ausführen sollen.
Der Bewegungsablauf eines einzelnen Beins bleibt vorgegeben, während die Koordination untereinander durch Interaktion des Modells mit der Umgebung erlernt werden soll. 
Diese Herangehensweise ermöglicht es die Komplexität der zu erlernenden Fähigkeiten zunächst zu reduzieren, lässt aber gleichzeitig die Möglichkeit offen, im weiteren Verlauf zusätzliche Elemente in den Lernprozess einzubeziehen.


\section{Verwandte Arbeiten}
\textbf{TODO: Ausformulieren !}
\begin{enumerate}

\item SMART-HexBot: a Simulation, Modeling, Analysis and
Research Tool for Hexapod Robot in Virtual Reality and
Simulink

\item Design, Simulation, and Control of a Hexapod Robot in
Simscape Multibody

\item Dynamic Modeling and Control of the Hexapod Robot Using Matlab SimMechanics

\end{enumerate}


\section{Zeitplan}
\begin{enumerate}
	\item Erstellen des "physischen" Hexapod-Modells in Simulink/Simscape + Testumgebung:
	\begin{enumerate}[label*=\arabic*.]
		\item Import eines 3D-Modells in die Simulink-Umgebung
		\item Konfigurieren eines "Inverse Kinematics Solvers" basierend auf der .urdf-Datei des 3D-Modells
		\item Definieren der Parameter des Roboters(Drehmomente der Motoren, Reibungskoeffizienten zwischen Beinen und Boden, max. Winkel, etc.)
		\item Erbauen einer einfachen Testumgebung
	\end{enumerate}
	\textbf{Geschätzte Dauer:} 2-3 Wochen
	
	\item Entwicklung des Bewegungscontrollers
	\begin{enumerate}[label*=\arabic*.]
		\item Bewegungsablauf eines einzelnen Beins
		\item Implementierung der Gangarten(Wave-, Tripod, etc.)	
		\item Erprobung und Optimierung der Bewegungen des Hexapods innerhalb der Simulationsumgebung
	\end{enumerate}
	\textbf{Geschätzte Dauer:} 3-4 Wochen
	
	\item Erlernen der Beinkoordination
	\begin{enumerate}[label*=\arabic*.]
		\item Aufstellen des Reinforcement Learning-Modells
		\item Beobachten des Lernprozesses, ggf. Anpassungen vornehmen
	\end{enumerate}
	\textbf{Geschätzte Dauer:} 4 Wochen
	

\end{enumerate}


Abschluss der Arbeit hoffentlich (Ende) September.