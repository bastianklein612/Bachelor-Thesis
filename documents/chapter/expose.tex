%!TEX root = ../thesis.tex
\chapter{Exposé}
\label{ch:expose}



\section{Introduction}
Simulations play an increasingly important role in the research and development of robots \parencite{afzal2020study}. 
They offer a cost-effective and efficient method of being able to investigate, analyze and optimize the behavior of robots even before they are physically implemented \parencite{en2019analysis}. 
Many companies involved in the field of robotics are also using the technology extensively in the development process of their systems, such as Boston Dynamics \parencite{BostonDynamicsSimulation}, Tesla Inc. \parencite{TeslaAiDay2022} or Kuka Robotics \parencite{KukaSim}.
Due to their high degree of mobility and versatility, hexapods are of particular interest in robotics. 
Compared to wheeled or bipedal robots, their six legs make it easier for them to overcome challenging terrain and complex obstacles \parencite{barai2013smart, atifystructure}.

\section{Problem}
The process of developing software for these hexapods is a challenging and by no means trivial task.
While maintaining balance is evidently easier for this architecture than for example the above mentioned bipedal robots, coordinating the many legs during movement is a major challenge \parencite{azayev2020blind,schilling2013walknet}.
To break down the complex dynamics of such a system, creating a virtual model can have many benefits.
There is a variety of robot simulation platforms, e.g. Gazebo, V-REP or Webots. 
Each of these simulators has its own strengths and features that researchers have utilized to explore different aspects of robotics.
Besides these tools explicitly developed for simulating robots, there are other modeling and simulation environments which have a wider fiend of use.
One example of these is MATLAB Simulink, a graphical programming environment used for modeling, simulation and analysis of complex, dynamic systems.
%Simulink's diagram-like interface and extensive library of "toolboxes" facilitate the creation of modular and well-organized systems, which can also be immediately simulated.
To support our research on hexapod robots, we aim to leverage simulation software to take advantage of the benefits these tools have to offer.
Considering the widespread use of MATLAB Simulink in both research and industry along with its extensive libraries addressing various domains, we have chosen it as our preferred platform.
%In our opinion, the MATLAB Simulink modeling and simulation environment from the MathWorks company is particularly interesting for implementing such a model.
%Due to its diagram-like development interface based on functional blocks and a large number of so-called "toolboxes" from a wide variety of application areas, Simulink enables the generation of modular and clearly arranged systems, which can be simulated immediately.
In recent years, several scientific papers have been published dealing with the simulation of a hexapod in Simulink(see e.g. \cite{tanaka2019development, barai2013smart, atify2019propelling}).
However, to our knowledge, there is currently no publicly available, up-to-date Simulink model on which further research could build on.

Therefore, in this bachelor thesis we want to address the problem of developing such a hexapod model using MATLAB Simulink.


%Auf Grund ihrer hohen Beweglichkeit und Vielseitigkeit sind Hexapods in der Robotik von besonderen Interesse. 
%Verglichen mit beräderten oder zweibeinigen Robotern können sie sich zuverlässiger auf unebenem Gelände bewegen und auch komplexe Hindernisse überwinden \parencite{barai2013smart, atifystructure}.
%Die Programmierung eines Bewegungscontrollers für solch einen Hexapod-Roboter ist eine anspruchsvolle, keinenfalls triviale Aufgabe.
%Obwohl das Aufrechterhalten der Balance bei einem Roboter mit 6 Beinen im Vergleich zu beispielsweise humanoiden Zweibeinern offensichtlich einfacher ist, stellt die Koordination der vielen Beine während der Fortbewegung eine große Herausforderung dar \parencite{azayev2020blind,schilling2013walknet}.
%Die Erzeugung eines virtuellen Modells kann hierbei viele Vorteile mit sich bringen.
%Es ermöglicht Designkonzepte zu testen, die Bewegung des Roboters zu analysieren und die Leistungsfähigkeit des Systems schon vor der physischen Umsetzung zu beurteilen \parencite{de2019analysis}.

%Besonders interessant für die Umsetzung eines solchen Modells ist unserer Meinung nach die Modellierungs- und Simulationsumgebung MATLAB Simulink des Unternehmens MathWorks.
%Durch die auf funktionalen Blöcken basierende, diagrammartige Entwicklungsoberfläche und eine Vielzahl sog. "Toolboxes" aus unterschiedlichsten Anwendungsgebieten ermöglicht Simulink die Erzeugung modularer und übersichtlicher Systeme, welche sofort simuliert werden können.
%In den letzten Jahren wurden mehrere wissenschaftliche Arbeiten veröffentlicht, die sich mit der Simulation eines Hexapods in Simulink beschäftigen(siehe z.B. \cite{tanaka2019development, barai2013smart, atify2019propelling}), sodass sich schließen lässt, dass die Vorteile der simulationsgestützten Entwicklung durchaus wahrgenommen und genutzt werden.
%Jedoch gibt es unseren Wissens nach derzeit kein öffentlich verfügbares, modernes Simulink-Modell auf dem weitere Forschung aufbauen könnte.

%Wir wollen uns in dieser Bachelorarbeit deshalb dem Problem der Entwicklung eines solchen Modells mit Hilfe von Simulink widmen.



%Um die Erforschung von Controllern zu Bewegungssteuerung möglichst effektiv vorantreiben zu können ist es wichtig, leistungsstarke Simulationssoftware in den Entwicklungsprozess einzubeziehen.
%Es wird ein wesentlich  kleinschrittiger Prozess ermöglicht, da jede noch so kleine Änderung direkt und ohne großen Mehraufwand getestet werden kann.

%Da auf Grund von Materialkosten, Konstruktions- und Wartungsaufwand reale Roboter oft diejenige Komponente eines Projektes sind, welche beim Erproben der neu entwickelten Software der größte zeitliche "bottleneck" sind, stellt das direkte Erproben komplexer Aufgaben in der realen Welt(z.B. Navigation auf unebenem Gelände)  ein Risiko dar.

%Zur Entwicklung des Controllers und der Simulation desselben auf einem Hexapod-Modell wird in dieser Arbeit das MATLAB Tool Simulink in Verbindung mit der Simulation-Toolbox SimScape verwendet werden.
%Simulink ist eine weit verbreitete Software zur Entwicklung und Simulation von komplexen dynamischen Systemen.
%Durch die Popularität der Software ist mit der Zeit eine Vielzahl von domänenspezifischen Bibliotheken(Toolboxes) entstanden, welche die einfache Modellierung verschiedenster Modelle ermöglichen.
%Der block-basierte Ansatz von Simulink ist dabei besonders hervorzuheben, da es dieser ermöglicht sehr modulare Systeme zu erzeugen, wodurch bei Anpassungen nur einzelne Subsysteme ausgetauscht werden, anstatt Änderungen am gesamten Modell vornehmen zu müssen.

\section{Ziel}
Das von uns im Rahmen dieser Bachelorarbeit angestrebte Ziel ist es, die Möglichkeiten von MATLAB Simulink zur Modellierung und Simulation komplexer dynamischer Systeme zu nutzen, um ein Modell eines Hexapod-Roboters sowie dessen Steuerung zu entwickeln.
Wir wollen mit den Ergebnissen dieser Arbeit die Vorteile eines durch Simulink gestützten Entwicklungsprozesses hervorheben und Impulse für eine ausgiebigere Nutzung derartiger Software setzen.

Das Modell muss die kinematischen Eigenschaften des Roboters und der Umgebung in welcher er sich bewegt berücksichtigen.
Es soll eine realistische Simulation des Roboterverhaltens ermöglichen und dabei Informationen bezüglich der Positionierung des Modells bereitstellen.
Diese Daten sind entscheidend für die Steuerung des Roboters und müssen für den Controller jederzeit zur Verfügung stehen.
Der Bewegungscontroller des Hexapods sollte in der Lage sein, den Roboter zuverlässig und präzise zu steuern.
Dies beinhaltet die Bewegung des einzelnen Beins, die Koordination der Beine untereinander sowie die Gleichgewichtserhaltung des Gesamtsystems.
Die entwickelte Lösung soll den Roboter in verschiedenen Gangarten geradlinig in der Simulationsumgebung bewegen und auch kleine Hindernisse überwinden können, ohne dass das System aus dem Gleichgewicht gebracht wird.
Dabei darf die Steuerung nur auf vom Modell bereitgestellte Daten zugreifen, wie dies auch bei einem realen Roboter mittels Sensoren der Fall wäre.
Es sollte außerdem möglich sein, Module des Bewegungscontrollers anpassen zu können oder Eigenschaften des Modells abzuändern, ohne das gesamte System neu konfigurieren zu müssen.
Nachdem ein robustes Modell und ein zuverlässiger Controller entwickelt wurden liegt der Fokus darauf, es dem Roboter mit Hilfe von maschinellem Lernen zu ermöglichen, die Koordination der Beine selbständig zu erlernen.
Es wäre besonders interessant festzustellen, ob schon bekannte Gangarten erkennbar werden oder ganz neue Formen der Bewegung entstehen.

\section{Lösungsansatz}
Die Grundlage der Arbeit liefert die in der Forschung und Industrie weit verbreite Software \textit{MATLAB\textregistered} und die enthaltene \textit{Simulink}\textregistered-Umgebung des Unternehmens \textit{MathWorks}.
Simulink bietet eine auf funktionalen Blöcken basierende Modellierungsmethode, welche einen sehr modularen und visuell übersichtlichen Entwicklungsprozess ermöglicht.
Durch eine Vielzahl von sogenannten "Toolboxes" können Modelle aus den verschiedensten Anwendungsgebieten erschaffen werden.

Zu Beginn der Arbeit wird ein detailliertes, physikalisch korrektes Modell des Hexapods mithilfe der Simulink-Toolbox Simscape entwickelt, als Vorlage dient hier der \emph{PhantomX}-Hexapod von \emph{Trossen Robotics}.
Jedes der 6 Beine besteht aus 3 Gelenken mit je einem Aktuator, eine Architektur welche weit verbreitet ist und sich bewährt hat.
Das Modell sollte möglichst gut parametrisiert werden, damit Eigenschaften wie Gewicht, Reibungskoeffizienten, Drehmomente, etc. der einzelnen Komponenten auch im nach hinein noch einfach angepasst werden können.
Jeder der Aktuatoren muss einzeln ansprechbar sein und die Gelenke müssen Sensordaten bezüglich ihrer Position (und Beschleunigung) liefern.
Diese Daten informieren den im nächsten Schritt zu implementierenden Controller über die Lage (und Dynamik) des Roboters.
Neben dem eigentlichen Modell des Hexapods wird zusätzlich noch eine einfache Testumgebung erbaut, in welcher das Robotermodell im weiteren Entwicklungsprozess ausgiebig getestet werden kann.

Nachdem die Entwicklung des SimScape-Modells abgeschlossen ist, wird mit der Implementierung einer kontrollierten Steuerung des Roboters in Simulink begonnen.
Der Fokus liegt dabei zunächst auf der Beschreibung des Bewegungsablaufs eines einzelnen Beins.
Sobald dieser Ablauf, bestehend aus Schwung- und Standphase, zuverlässig funktioniert, wird er auf alle 6 Beine ausgeweitet.
Hierbei spielt die Koordination der Beine eine entscheidende Rolle, unabhängig davon, ob diese zentral oder verteilt ausgelegt wird.
Grundlegend ist es natürlich auch, dass der Roboter das Gleichgewicht hält und robust gegenüber Störungen im Bewegungsablauf ist.
Um das SimScape-Modell ausgiebig zu testen, ist es möglich, mehrere verschiedene Gangarten wie den Tripod-, Tetrapod- und Wave-Gait abzubilden.
%evt. könnte auch eine dezentrale Architektur wie z.B. die von Schilling et al. abgebildet werden (?).

Basierend auf den Vorarbeiten zur Modellierung und den verschiedenen Gangarten soll anschließend die Koordination der Beine mithilfe eines Reinforcement Learning-Modells selbst erlernt werden.
Dieser Lernprozess wird ebenfalls innerhalb der Simulink-Umgebung durch die RL-Toolbox realisiert.
Im Gegensatz zu den bisher erzeugten Gangarten wird hier nicht vorgegeben werden, welche Beine ihren Bewegung zu welchem Zeitpunkt ausführen sollen.
Der Bewegungsablauf des einzelnen Beins bleibt fest definiert, während die Koordination untereinander durch Interaktion des Modells mit der Umgebung erlernt werden soll. 
Diese Herangehensweise ermöglicht es die Komplexität der zu erlernenden Fähigkeiten zunächst zu reduzieren, lässt aber gleichzeitig die Möglichkeit offen, im weiteren Verlauf zusätzliche Elemente in den Lernprozess einzubeziehen.


\section{Verwandte Arbeiten}

Im Bereich der Simulation von Hexapod-Roboter wurden in den letzten Jahren zahlreiche wissenschaftliche Arbeiten veröffentlicht, auch spezifisch unter Verwendung der Software Simulink.
So beschäftigt sich \parencite{HexapodRobotSimscapeMultibody} mit der Umsetzung eines Hexapods aus einfachen geometrischen Figuren und dessen Steuerung in Simulink.

\parencite{DynamicModelingAndControlMatlabSimMechanics} nutzen ein detailierteres CAD-Modell, welches dem hier verwendeten Roboter sehr nahe kommt und kommen so dem Testen auf einem realen Roboter sehr nahe.

\textbf{TODO: Erweitern und überarbeiten !!}


\section{Zeitplan}
\begin{enumerate}
	\item \textbf{Erstellen des "physischen" Hexapod-Modells in Simulink/Simscape + Testumgebung:}
	\begin{enumerate}[label*=\arabic*.]
		\item Import eines 3D-Modells in die Simulink-Umgebung
		\item Konfigurieren eines "Inverse Kinematics Solvers" basierend auf der .urdf-Datei des 3D-Modells
		\item Definieren der Parameter des Roboters(Drehmomente der Motoren, Reibungskoeffizienten zwischen Beinen und Boden, max. Winkel, etc.)
		\item Erbauen einer einfachen Testumgebung
	\end{enumerate}
	\textbf{Geschätzte Dauer:} 2-3 Wochen
	
	\item \textbf{Entwicklung des Bewegungscontrollers:}
	\begin{enumerate}[label*=\arabic*.]
		\item Bewegungsablauf eines einzelnen Beins
		\item Implementierung der Gangarten(Wave-, Tripod, etc.)	
		\item Erprobung und Optimierung der Bewegungen des Hexapods innerhalb der Simulationsumgebung
	\end{enumerate}
	\textbf{Geschätzte Dauer:} 3-4 Wochen
	
	\item \textbf{Erlernen der Beinkoordination:}
	\begin{enumerate}[label*=\arabic*.]
		\item Aufstellen des Reinforcement Learning-Modells
		\item Beobachten des Lernprozesses, ggf. Anpassungen vornehmen
	\end{enumerate}
	\textbf{Geschätzte Dauer:} 4 Wochen
	

\end{enumerate}


Abschluss der Arbeit hoffentlich (Ende) September.