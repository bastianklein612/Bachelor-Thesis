%!TEX root = ../thesis.tex
\chapter{Exposé}
\label{ch:expose}



\section{Einleitung}
Der Einsatz von Simulationen spielt in der Erforschung und Entwicklung von Robotern eine zunehmend wichtige Rolle. 
Sie bieten eine kostengünstige und effiziente Methode, das Verhalten von Robotern schon vor der physischen Umsetzung untersuchen, analysieren und optimieren zu können. 
%Es lassen sich komplexe Bewegungsabläufe und verschiedenste Szenarien schnell und sicher erproben, ohne dass dabei das Risiko von Schäden an einem realen System in Kauf genommen werden müsste.

Auf Grund ihrer hohen Beweglichkeit sowie Vielseitigkeit sind Hexapods in der Robotik von besonderen Interesse. 
Verglichen mit beräderten oder zweibeinigen Robotern können sie sich wesentlich zuverlässiger auf unebenem Gelände bewegen und auch komplexe Hindernisse überwinden.
Die Simulation eines solchen Hexapods ermöglicht es sein Verhalten in verschiedensten Situationen zu erproben und die Kontroll-Algorithmen zu optimieren, noch bevor der Roboter physisch konstruiert wird.

\section{Problem}

Die Entwicklung und Erprobung von Robotern und ihren Bewegungsabläufen ist zeitaufwändig und mit hohen Kosten verbunden.
Dabei kann die Beschädigung von einem der oft wenigen und teuren Exemplare, durch beispielsweise einen fehlerhaft agierenden Bewegungscontroller, den Entwicklungsprozess verzögern und zusätzliche Materialkosten erzeugen.
Auch ist ein iterativer Optimierungsprozess welcher ausschließlich in einer realen Umgebung stattfindet langsamer, als wenn dieser mit Hilfe von Simulationen unterstützt wird.
Die Steuerungssoftware muss für jede Anpassung neu auf die Hardware übertragen werden und Testaufbau sowie Roboter müssen manuell wieder in den Ausgangszustand versetzt werden.
Ebenfalls kann es schwierig sein fehlerhaftes Verhalten, welches beim Test mit einem realen System beobachtet wurde, zu reproduzieren ohne die Möglichkeit zu haben den Testaufbau identisch wiederherzustellen.

Natürlich ist es letztendlich wichtig, den entwickelten Controller auch in der echten Welt zu testen, damit sichergestellt werden kann, dass dieser auch unter den tatsächlichen physikalischen Begebenheiten zuverlässig funktioniert.
Die Vorteile welche eine Simulationsumgebung mit sich bringt sollten jedoch keinesfalls übersehen werden, da diese für Entwickler und Forscher ein mächtiges Werkzeug darstellen kann um vor der realen Erprobung erste Tests durchzuführen, Fehler aufzudecken und den Controller zu verbessern.
%Um die Erforschung von Controllern zu Bewegungssteuerung möglichst effektiv vorantreiben zu können ist es wichtig, leistungsstarke Simulationssoftware in den Entwicklungsprozess einzubeziehen.
%Es wird ein wesentlich  kleinschrittiger Prozess ermöglicht, da jede noch so kleine Änderung direkt und ohne großen Mehraufwand getestet werden kann.

%Da auf Grund von Materialkosten, Konstruktions- und Wartungsaufwand reale Roboter oft diejenige Komponente eines Projektes sind, welche beim Erproben der neu entwickelten Software der größte zeitliche "bottleneck" sind, stellt das direkte Erproben komplexer Aufgaben in der realen Welt(z.B. Navigation auf unebenem Gelände)  ein Risiko dar.

%Zur Entwicklung des Controllers und der Simulation desselben auf einem Hexapod-Modell wird in dieser Arbeit das MATLAB Tool Simulink in Verbindung mit der Simulation-Toolbox SimScape verwendet werden.
%Simulink ist eine weit verbreitete Software zur Entwicklung und Simulation von komplexen dynamischen Systemen.
%Durch die Popularität der Software ist mit der Zeit eine Vielzahl von domänenspezifischen Bibliotheken(Toolboxes) entstanden, welche die einfache Modellierung verschiedenster Modelle ermöglichen.
%Der block-basierte Ansatz von Simulink ist dabei besonders hervorzuheben, da es dieser ermöglicht sehr modulare Systeme zu erzeugen, wodurch bei Anpassungen nur einzelne Subsysteme ausgetauscht werden, anstatt Änderungen am gesamten Modell vornehmen zu müssen.

\section{Ziel}
Das angestrebte Ziel dieser Bachelorarbeit ist es, einen Hexapod-Roboter und dessen Steuerung innerhalb der MATLAB Simulink Umgebung zu entwickeln.
Konkret soll dies die Modellierung des physischen Modells mit Hilfe von SimScape, den Aufbau eines Controllers für die Bewegungssteuerung des Roboters sowie das Erlernen der Koordination der Beine untereinander mittels Reinforcement Learning.

\begin{enumerate}
	\item \textbf{Modellierung des Hexapods + Umgebung:} Es wird ein physikalisch korrektes Modell eines Hexapod-Roboters mit Hilfe der Simulink Toolbox SimScape erschaffen.
	Dies soll auch eine einfache Simulationsumgebung beinhalten, in welcher das Modell getestet werden kann. 
	
	\item \textbf{Entwicklung eines Bewegungscontrollers:} Der Hexapod soll durch einen Bewegungscontroller in die Lage versetzt werden, sich (geradlinig) durch eine simulierte Umgebung zu bewegen. 
	Der Controller wird in Simulink implementiert und verwendet die Informationen aus der Simulation um den Roboter präzise zu steuern.
	Er muss die Bewegung der einzelnen Beinsegmente sowie die Koordination der Beine untereinander steuern können und dabei auch auf das Gleichgewicht des Gesamtsystems achten.
	
	\item \textbf{Selbstständiges Erlernen der Beinkoordination:} Durch Interaktion mit der Simulationsumgebung soll das Modell selbstständig die Koordination der 6 Beine erlernen, die Beinbewegung selbst bleibe dabei vorgegeben.
	Modelliert werden soll der Lernprozess mit Hilfe der Reinforcement Learning Toolbox von Simulink.
	
\end{enumerate}

\section{Lösungsansatz}
Zu Beginn der Arbeit muss ein vereinfachtes, jedoch physikalisch korrektes Modell des realen Hexapods mit der Simulink-Toolbox Simscape entwickelt werden.
Dieses sollte möglichst gut parametrisiert sein, damit Eigenschaften wie Dimension, Gewicht, Reibungskoeffizienten, Drehmomente, etc. der einzelnen Bestandteile auch im nach hinein noch einfach angepasst werden können.
Jedes der 6 Beine des Roboters soll aus drei Gelenken mit jeweils einem Aktuator bestehen.
Diese Bein-Architektur ist bei Hexapods weit verbreitet und hat sich bewährt.
Jeder der Aktuatoren sollte einzeln ansprechbar sein, die Gelenke sollten Sensordaten bezüglich Position (und Beschleunigung) liefern.
Diese Daten informieren den im nächsten Schritt zu implementierenden Controller über die Lage (und Dynamik) des Roboters 

Nachdem die Entwicklung des SimScape-Modells abgeschlossen ist, wird mit der Implementierung einer kontrollierten Steuerung des Roboters in Simulink begonnen.
Der Fokus liegt dabei zunächst auf der Beschreibung des Bewegungsablaufs eines einzelnen Beins.
Sobald dieser Ablauf, bestehend aus Schwung- und Standphase, zuverlässig funktioniert, wird er auf alle 6 Beine ausgeweitet.
Hierbei spielt die Koordination der Beine eine entscheidende Rolle, unabhängig davon, ob diese zentral oder verteilt ausgelegt wird.
Um das SimScape-Modell ausgiebig zu testen, ist es möglich, mehrere verschiedene Gangarten wie den Tripod-, Tetrapod- und Wave-Gait abzubilden.
%evt. könnte auch eine dezentrale Architektur wie z.B. die von Schilling et al. abgebildet werden (?).

Basierend auf den Vorarbeiten zur Modellierung und den verschiedenen Gangarten soll anschließend die Koordination der Beine mithilfe eines Reinforcement Learning-Modells selbst erlernt werden.
Dieser Lernprozess wird ebenfalls innerhalb der Simulink-Umgebung durch die RL-Toolbox realisiert.
Im Gegensatz zu den bisher erzeugten Gangarten wird hier nicht vorgegeben, welche Beine ihren Bewegung zu welchem Zeitpunkt ausführen sollen.
Der Bewegungsablauf eines einzelnen Beins bleibt vorgegeben, während die Koordination untereinander erlernt werden soll. 
Diese Herangehensweise ermöglicht es die Komplexität der zu erlernenden Fähigkeiten zunächst zu reduzieren, lässt aber gleichzeitig die Möglichkeit offen, im weiteren Verlauf zusätzliche Elemente in den Lernprozess einzubeziehen.


\section{Verwandte Arbeiten}
\begin{enumerate}

	

\item SMART-HexBot: a Simulation, Modeling, Analysis and
Research Tool for Hexapod Robot in Virtual Reality and
Simulink

\item Design, Simulation, and Control of a Hexapod Robot in
Simscape Multibody

\item Dynamic Modeling and Control of the Hexapod Robot Using Matlab SimMechanics

\end{enumerate}


\section{Zeitplan}
\begin{enumerate}
	\item Entwicklung des "physischen" Hexapods in Simulink/Simscape + Testumgebung: 1-2 Wochen.
	\item Erzeugung des Bewegungsabläufe für Wave-, Tripod- und evt. dezentrale Architektur 5-6 Wochen (?).
	\item Nutzung von Reinforcement Learning zum Erlernen der vorher sequenziell gesteuerten Beinbewegungen 4 Wochen (?).
\end{enumerate}


Abschluss der Arbeit hoffentlich (Ende) September.