%!TEX root = ../thesis.tex
\chapter{Related Works}
\label{ch:relatedWorks}

In recent years, numerous scientific papers dealing with the simulation of hexapod locomotion have been published, also specifically targeting the MATLAB Simulink platform.

Using a hexapod model very similar to ours, Beaber et al. \parencite{beaber2018dynamic} developed a complete dynamic model of a hexapod robot supported by the Simulink toolbox SimMechanics(now Simscape), concentrating on generating a tripod gait.

Thilderkvist \& Svensson \Parencite{thilderkvist2015motion} applied a Model-Based Design process utilizing both a virtual model and physical hexapod robot to develop a motion controller.
Their work emphasized the value of Simulinks simulation capabilities and that of the integrated code generation during the development process.

Exploring the utilization of Reinforcement Learning to control the robots movements, Trotta et al. \parencite{trotta2022walking} aimed to develop the foundational architecture and control of a hexapod to be used for space exploration.
They highlight the challenges faced during the RL learning process, particularly the models tendencies to exploit reward function constraints, leading to the emergence of undesirable behaviors.
The study points out the potential of RL for enhancing hexapod control but also shows the difficulty of its implementation.


\todo{This is currently just copied from the expose; EXPAND and REWRITE}