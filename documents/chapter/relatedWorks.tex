%!TEX root = ../thesis.tex
\chapter{Related Works}
\label{ch:relatedWorks}

In recent years, numerous scientific papers dealing with the simulation of hexapod locomotion and reinforcement learning have been published.
Some of these papers specifically focus on utilising MATLAB Simulink, while others are based on different simulators or real hardware but deal with similar topics.

Using a hexapod model similar to ours, Beaber et al. \parencite{beaber2018dynamic} develop a dynamic model of a hexapod robot supported by the Simulink toolbox SimMechanics (predecessor to Simscape), concentrating on the generation of  a tripod gait.
By creating a complete forward and inverse kinematics model of the robot they are able to generate individual leg trajectories, divided into support (stance) and transfer (swing) phase, in a way that allows the robot to walk in a straight path.
They explicitly highlight the simplicity of developing in Simulink compared to traditional modelling tools.

Thilderkvist \& Svensson \Parencite{thilderkvist2015motion} apply a model-based design process utilizing both a virtual model and physical hexapod robot to develop a motion controller.
They demonstrate Simulinks usefulness for testing control algorithms before transferring them onto the hardware.
Their work emphasizes the value of Simulinks simulation capabilities and that of the integrated code generation/ hardware integration during the development process.
They also report non-satisfying results concerning contact-force modelling using SimMechanics.
%We attribute these to be most-likely linked to an earlier MATLAB version used for their work.

Exploring the use of Reinforcement Learning for movement control, Trotta et al. \parencite{trotta2022walking} develop the foundational architecture and control of a hexapod to be used for space exploration.
Using a simplified hexapod model, they train an RL agent on the task of walking on flat terrain.
Due to limited computational resources, they reduce the agents action space to just 3 elements, the amplitude and the phase differences between the opposing tripods, which results in, as they pre-emptively predict, the agent learning a, although not perfect, tripod gait.
They highlight the challenges faced during the training process, particularly the models tendencies to exploit reward function constraints, leading to the emergence of undesirable behaviour.
The study points out the potential of RL for enhancing hexapod control but also show the difficulty of its implementation.
