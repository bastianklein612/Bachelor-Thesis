%!TEX root = ../thesis.tex
\chapter{Results}
\label{ch:results}

\section{intro}
Broad introduction to results chapter.




\section{Simulink model review}
During the development process we observed that the initially created model can be easily modified.
As an example, additional sensory output was added without any complications.
The sensor just needs to be added to the model and the signal routed through the already existing sonsory information bus.

Given a moderately small step size of 0.5ms(in some cases 0.25ms) the model can be simulated without running into any Simscape exceptions.
The simulation speed is close to real time given 1 core and 0.5ms step-size.
If the chosen step size is to large(>0.5ms) the simulation becomes unstable and can run into exceptions.
These most often occur due to objects moving into each other faster that the solver can observe due to its low time resolution and thus cause collision exceptions.

The robots behavior seems, to the best of our knowledge and without examining a real world model, physically accurate.





\section{RL Learning review}

Trying to learn the coordination of a hexapod robots legs without any significant prior knowledge of reinforcement learning proved to be more difficult than anticipated.
Even though RL could be described as a black box with in- and outputs which learns automatically what to do given a well defined reward function, there is an overwhelming amount of parameters to consider.
Which reinforcement learning algorithm should be chosen, which learning parameters need to be changed, which and how many layers should the NN consist of, how many episodes of training are needed to see results ?


Early on in the RL setup process we realized that the current inputs into the hexapod model are not suitable for reinforcement learning.
These initial inputs, frequency, duty cycle and offsets were better suited to being statically optimized instead of learned.
The new inputs, one signal per leg to initiate the swing phase, were much more suited for RL, as they enable a more complex and dynamic interaction with the robot.
Only after implementing these changes were we able to record the first successes in learning.


\textit{MATLABs Parallel Computing Toolbox} provided a significant boost in learning performance.
By being able to run 8 agents in parallel , one per processor core, we were able to speed up the learning process by a factor of 8.
This allowed us to test several different agent configurations in an acceptable amount of time and also enabled us to refine well performing agents by running more episodes.

The simulation studies and RL training performed in this work are accomplished with a desktop computer powered by Intel i7-11700K CPU, 32 GB of RAM and NVIDIA RTX 2060 GPU.

Concerning MATLAB, we used the version 2023a for all of our work.


