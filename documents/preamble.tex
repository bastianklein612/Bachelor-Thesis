%!TEX root = thesis.tex
% Author: Jannes Bantje
% Modified: Lars Haalck lars.haalck@wwu.de
% please ask Lars Haalck first if you have any questions

\documentclass[%
a4paper,
parskip=half,
index=totoc,
toc=listof,
fontsize=11,
headinclude,
oneside,    %changed from 'twoside'
BCOR=12mm,
cleardoublepage=empty,
DIV=13,
%draft,
final
]{scrreprt}

\usepackage[usenames,x11names]{xcolor}
\usepackage[final]{graphicx}
\usepackage{subcaption}
\usepackage{datetime}

% typographic settings, fonts, and math
\usepackage[utf8]{inputenc}
\usepackage[semibold]{libertine}
\usepackage[T1]{fontenc}
\usepackage{textcomp} % verhindert ein paar Fehler bei den Fonts
\usepackage[varl]{zi4}
\usepackage{mathtools,amssymb,amsthm} % Verbesserung von th (die amsmath selbst lädt)
\usepackage[libertine,cmintegrals,bigdelims,varbb]{newtxmath}
\usepackage[english]{babel}
\usepackage[babel=true, tracking=true,final]{microtype}

%greek letters in text without changing to math mode
\usepackage{textgreek}

%scalable vector graphics(svg)
\usepackage[inkscapeformat=png]{svg}

%multi-line comments
\usepackage{verbatim}

%prevents images from appearing in a different section than where they are actually placed
\usepackage[section]{placeins}

%Modify figure and table captions
\usepackage[font=small,labelfont=bf,labelsep=space]{caption}
\captionsetup{%
	figurename=Figure,
	tablename=Table.,
	justification=centerlast,
	singlelinecheck=false,
	format=hang,
	font=footnotesize,
	labelfont=bf
}


% set line spacing
\usepackage{setspace}
% for example 1.5 line spacing
\onehalfspacing

% literature settings
\usepackage[%
backend=biber,
style=apa,
sortlocale=auto,
natbib,
hyperref,
backref,
mincitenames=1,
maxcitenames=1
]%
{biblatex}
\addbibresource{sources.bib} % sets literature file

% hyperref settings to make links clickable in PDF
\usepackage[%
hidelinks,
pdfpagelabels,
bookmarksopen=true,
bookmarksnumbered=true,
linkcolor=orange,
urlcolor=orange,
plainpages=false,
pagebackref,
citecolor=orange,
hypertexnames=true,
pdfborderstyle={/S}, %   /U hinzufügen für unterstrichene Links
colorlinks=true,
backref=false,
pdfencoding=auto,
psdextra
]{hyperref}
\hypersetup {final}


% enumeration settings
\usepackage[shortlabels]{enumitem}
\setlist[enumerate,description]{font=\sffamily\bfseries} % makes labels in enumeration bold
\usepackage{csquotes}

\usepackage{ifdraft}
\setlength{\marginparwidth}{2.0cm}
\ifoptionfinal{}{
    % enable this line to better visualize overflows
    % \PassOptionsToPackage{showframe}{geometry}

    \paperwidth=\dimexpr \paperwidth + 3cm\relax
    \oddsidemargin=\dimexpr\oddsidemargin + 0cm\relax
    \evensidemargin=\dimexpr\evensidemargin + 3cm\relax
    \setlength{\marginparwidth}{2.5cm}
}

\usepackage[pass]{geometry}
% allows adding of todo notes at the side of the document
\usepackage[textsize=small,textwidth=2.5cm]{todonotes}

%enables easy dispalying of 1e-4, ...
\usepackage{siunitx}

% settings for the header and footer
\usepackage[headsepline=1pt]{scrlayer-scrpage}
\pagestyle{scrheadings}
\clearpairofpagestyles % clear defaults
\setkomafont{headsepline}{\color{gray}} % adds a gray line under the header

% set section title on the right, and chapter title on the left page in a double page
% document
\automark[section]{chapter}

\rohead{\rightmark} % section title on the right side
\lehead{\scshape\leftmark} % chapter title on the left side an in small caps
\ofoot[\pagemark]{\pagemark} % page marks always on the outer site of the page

% sets page marks and footer and header texts to sans-serif in gray
\renewcommand*{\pnumfont}{\sffamily}
\renewcommand*{\footfont}{\sffamily\color{gray}}
\renewcommand*{\headfont}{\sffamily\color{gray}}

% change the chapter, section and subsection font sizes and spacings a bit for a5 format
% \setlength{\footskip}{1.75\baselineskip} % change the spacing a bit for a5 format
% \RedeclareSectionCommand[%
% afterskip=1\baselineskip,%
% beforeskip=-1\baselineskip]{chapter}

% \setkomafont{chapter}{\LARGE}
% \setkomafont{section}{\Large}
% \setkomafont{subsection}{\large}

% adds a thick gray line after the chapter number
\renewcommand*{\chapterformat}{%
    \thechapter\enskip
    \textcolor{gray!50}{\rule[-\dp\strutbox]{1.5pt}{\baselineskip}}\enskip
}

%creates a subsection without adding it to the table of contents
%section keeps subsection number(* does not do this)
\newcommand{\hiddensubsection}[1]{
	\stepcounter{subsection}
	\subsection*{\arabic{chapter}.\arabic{section}.\arabic{subsection}\hspace{1em}{#1}}
}

% math environments
\usepackage{amsthm}
\usepackage{thmtools}
\usepackage{mdframed}
\usepackage{blindtext}
\renewcommand{\listtheoremname}{Übersicht aller Aussagen}

% -- Theoreme als PDF-Lesezeichen
\usepackage{bookmark}
\bookmarksetup{open,numbered}
\makeatletter
\newcommand*{\theorembookmark}{%
   \bookmark[
     dest=\@currentHref,
     rellevel=1,
     keeplevel,
   ]{%
     \thmt@thmname\space\csname the\thmt@envname\endcsname
     \ifx\thmt@shortoptarg\@empty
     \else
       \space(\thmt@shortoptarg)%
     \fi
   }%
}
\makeatother

% -- Definition der einzelnen Umgebungen
\declaretheoremstyle[%
     headfont=\sffamily\bfseries,
     notefont=\normalfont\sffamily,
     bodyfont=\normalfont,
     headformat=\NAME\ \NUMBER\NOTE,
     headpunct=,
     postheadspace=\newline,
     spaceabove=\parsep,spacebelow=\parsep,
     %shaded={bgcolor=gray!20},
     postheadhook=\theorembookmark,
     mdframed={
         backgroundcolor=gray!20,
             linecolor=gray!20,
             innertopmargin=6pt,
             roundcorner=5pt,
             innerbottommargin=6pt,
             skipbelow=\parsep,
             skipbelow=\parsep }
     ]%
{mainstyle}

\declaretheoremstyle[%
     headfont=\sffamily\bfseries,
     notefont=\normalfont\sffamily,
     bodyfont=\normalfont,
     headformat=\NAME\ \NUMBER\NOTE,
     headpunct=,
     postheadspace=\newline,
     spaceabove=15pt,spacebelow=10pt,
     postheadhook=\theorembookmark]%
{mainstyle_unshaded}

\declaretheoremstyle[%
     headfont=\sffamily\bfseries,
     notefont=\normalfont\sffamily,
     bodyfont=\normalfont,
     headformat=\NUMBER\NAME\NOTE,
     headpunct=,
     postheadspace=\newline,
     spaceabove=15pt,spacebelow=10pt,
     % shaded={bgcolor=gray!20},
     postheadhook=\theorembookmark]%
{mainstyle_unnumbered}

\declaretheorem[name=Definition,parent=section,style=mainstyle]{definition}
\declaretheorem[name=Definition,numbered=no,style=mainstyle]{definition*}
\declaretheorem[name=Definition,sharenumber=definition,style=mainstyle_unshaded]{definitionUnshaded}

\declaretheorem[name=Theorem,sharenumber=definition,style=mainstyle]{theorem}
\declaretheorem[name=Theorem,numbered=no,style=mainstyle_unnumbered]{theorem*}

\declaretheorem[name=Proposition,sharenumber=definition,style=mainstyle]{proposition}
\declaretheorem[name=Lemma,sharenumber=definition,style=mainstyle]{lemma}

\declaretheorem[name=Satz,sharenumber=definition,style=mainstyle]{satz}
\declaretheorem[name=Satz,sharenumber=definition,style=mainstyle_unshaded]{satzUnshaded}
\declaretheorem[name=Satz,numbered=no,style=mainstyle_unnumbered]{satz*}

\declaretheorem[name=Korollar,sharenumber=definition,style=mainstyle]{korollar}

\declaretheorem[name=Notation,numbered=no,style=mainstyle_unnumbered]{notation}
\declaretheorem[name=Bemerkung,numbered=no,style=mainstyle_unnumbered]{bemerkung}
\declaretheorem[name=Beispiel,numbered=no,style=mainstyle_unnumbered]{beispiel}
\declaretheorem[name=Beispiele,numbered=no,style=mainstyle_unnumbered]{beispiele} 
